\documentclass{article} % Choose the document class
\usepackage{amsmath} % For math support, if needed
\usepackage{graphicx} % For including images, if needed
\usepackage[backend=biber]{biblatex}
\usepackage{markdown}
\addbibresource{references.bib}
\title{Designing a general abstract configuration format for Geoscience Modeling and Simulation}
\author{Lyssandre Selene} 
\date{\today}

\begin{document} % Start of the document

\maketitle % Generates the title

\section{Introduction} % Section heading
Ok so here my research starts, god saves the queen.

My goal is to find an abstract way to define a geoscience simulation modelling problem,
where all implementations details are abstracted.

Therefore, I need to make a review on what kinds of problems geoscientists handle and how they describe them.

\section{Geoscience Problems Review}
More content here. 

\subsection{Reservoir Exploitation}
Reservoir exploitation includes for example carbon sequestration or oil production.

Defining a reservoir simulation problem is all about listing fluid and reservoirs properties, 
including their relationship to pressure, temperature and composition. \cite{reservoir-review}


\paragraph{Reservoir properties}
\begin{itemize}
  \item fluid 
    \begin{itemize}
      \item yo
    
  \end{itemize} 
\end{itemize}

\begin{description}
  \item[phase ($a$)] region of space with uniform physical properties
  \item[fluid phase]
  \item[liquid aqueuos phase (w)]
  \item[liquid oleic phase (o)]
  \item[gaseous phase (g)]
  \item[microemulsions]
  \item[supercritical fluids]
  \item[rock phase] rock matrix
  \item[component (k)] unique chemical species, many components can be in one phase and a component can be in several phases
  \item[pseudocomponent] a pseudocomponent represents similar components \\ lumped together for computational purporses, simulation is often based on three to ten pseudocomponents
  \item[compositional model]
  \item[black oil model ($\beta$)] a special case of compositional model, with three pseudocomponents (water, oil and gas) in up to three fluid phases (aqueous, oleic and gaseous)

    The oil pseudocomponent consists of all componenents present in a liquid hydrocarbon phase when brought to standard conditions. Likewise,
  \item[standard conditions (sc)] $T_{sc}$ = 15.5 °C  $p_{sc}$ 0.95 bar
  \item[porosity ($\varphi$)] pore volume divided by bulk volume
  \item[phase saturation ($S_{\alpha}$)] volume fraction of the void space occupied by a phase, phase saturations must sum to 1
  \item[density ($\rho$)] mass divided by volume, usually refers to a phase but can also apply to a component. The density of a fluid is a function of pressure, temperature and composition.
\end{description}



% \begin{markdown}
% ### Hey man
% ### Definition
% phase :
% - god that was hard
%   - fazfa
%
%
% \end{markdown}


\subsection{Geothermal Energy}


\printbibliography
\end{document} % End of the document

